%% mdsThesis
%%   sphKernelPlot.tex

\documentclass{standalone}

\usepackage{mdsTikz}

%% mdsThesis
%%   sphFunctions.tex

% Gaussian
\pgfmathdeclarefunction{gaussian}{2}{
  \pgfmathparse{1/(#2*sqrt(2*pi))*exp(-((x-#1)^2)/(2*#2^2))}%
}

%% plot parameters
\providecommand{\polarViewAngle}{110} % polar camera angle in degrees
\providecommand{\azimuthViewAngle}{110} % azimuthal camera angle in degrees


\begin{document}

  \tdplotsetmaincoords{\polarViewAngle}{\azimuthViewAngle}
  \begin{tikzpicture}[tdplot_main_coords, scale = .5]
	% Draw grid
    \foreach \x in {-8,-7,...,1}
      \draw[dashed,MUN_Grey] (-8,\x,0) -- (6,\x,0);
    \foreach \y in {-8,-7,...,6}
      \draw[dashed,MUN_Grey] (\y,-8,0) -- (\y,1,0);
    % Draw SPH particles
    \begin{scope}[canvas is yx plane at z=0]
      \fill[fill=\halfgrey, draw=\halfgrey, opacity=0.7] 
        (-8,-8) rectangle (1,6) 
        node[right,\halfgrey] {$\Omega \subseteq \mathbb{E}^2$};
      % Draw origin
      \fill[fill=black] 
        (3pt,3pt-6.5cm) arc (0:360:3pt) 
        node[below,\halfgrey] {$\text{O}$};
      % Support domain of particle a
      \fill[fill=MUN_Red, draw=MUN_Grey, opacity=0.4] 
        (-0.5,0) arc (0:360:3cm) 
        node[left,xshift=-2.5cm,black] {$\Omega_{\br}$};
      % Position r_a
      \draw[black,->] (0,3pt-6.5cm) -- (3pt-3.5cm,0);
      \fill[fill=MUN_Red,draw=black]
        (3pt-3.5cm,0) arc (0:360:3pt) 
        node[below,black] {$\mathbf{r}$};
      % Support domain radius
      \draw[black,->,opacity=0.7] 
        (-3.5,0) -- (2.12132-3.5,2.12132) 
        node[below,black] {$\kappa h$};
    \end{scope}
    % Draw wh-axes
    \draw[->] (0,-3.5,0) -- (0,-3.5,5.5) node[left,\halfgrey]{$w_h(\mathbf{r})$};
    % Draw w_h(r_a)
    \begin{scope}[canvas is yz plane at x=0]
      \begin{axis}[
        no markers, clip=false,
        xshift=-7.1cm, yshift=-.5cm,
        width=8.75cm,
        axis y line=none, axis x line=none,
        domain=-1.5:1.5, samples=100,
        xtick=\empty, ytick=\empty
        ]
        \addplot[color=MUN_Red,very thick] {gaussian(0,0.5)};
      \end{axis}
    \end{scope}
  \end{tikzpicture}
\end{document}